\chapter{Introduction}{\label{ch:intro}}
\section{Background}

The integration of Virtual Reality (VR) technologies in the field of driver training has opened new avenues for creating realistic and safe learning environments. Traditional driving instruction often faces limitations due to safety concerns, vehicle costs, and environmental constraints. Additionally, training on high-end or luxury vehicles such as Mercedes-Benz, BMW, or Audi models is often impractical for driving schools because of their operational and maintenance costs. To overcome these challenges, VR-based driving simulators have emerged as an effective alternative, allowing learners to experience authentic driving conditions without the associated risks or expenses \cite{Harari2021}.

With advancements in modern VR hardware such as the Meta Quest 3, it has become feasible to develop highly immersive simulations that closely replicate real-world vehicle dynamics, visual fidelity, and driver feedback. This project aims to design and develop a \textbf{VR-based Driving Simulator for Driving Schools}, focusing on high-end vehicle simulations. The system will leverage custom-built input hardware—including a rewired steering wheel, accelerator, brake, and clutch assembly—connected via Bluetooth to the simulation platform. These components will provide realistic control feedback and allow for seamless integration with both automatic and manual vehicle modes.

\section{Problem Statement}

Conventional driving simulators are often expensive, bulky, and lack realistic tactile feedback from vehicle controls. Furthermore, most commercially available simulators are targeted at entertainment rather than driver education, leading to a gap in the availability of accurate and affordable driver training systems. There exists a need for a modular, scalable, and hardware-accurate VR driving simulator that can be customized for specific vehicle models and used effectively in a driving school setting.

\section{Objectives}

The primary objective of this project is to develop a patentable, high-fidelity VR driving simulator that provides an immersive and realistic driving experience. The specific goals include:
\begin{itemize}
    \item Designing a VR-compatible driving simulation system for Meta Quest 3 and PC platforms.
    \item Developing realistic vehicle physics and environmental models using Unity or Unreal Engine.
    \item Integrating physical driving controls (steering, pedals, gear shifter) with Bluetooth-based input communication.
    \item Providing accurate feedback mechanisms and data logging for performance assessment.
    \item Creating an extensible system architecture that allows for both automatic and manual transmission modes.
\end{itemize}

\section{Scope}

The system will initially focus on simulating automatic vehicles, with the potential extension to manual transmission as a secondary phase. The prototype will be tested using a Meta Quest 3 headset connected to a PC equipped with an NVIDIA RTX 3060 GPU. The developed system will be optimized for standalone operation on Meta Quest devices, with scalability for PC-VR tethered mode. In the long term, the simulator can be expanded for collaboration with government agencies such as the Mumbai RTO for standardized driver education and testing.

\section{Report Overview}

This report presents the design, implementation, and evaluation of a VR-based driving simulator that integrates realistic physical controls with an immersive headset platform. The document first explains the motivation and technical background for the project, then presents the system architecture, hardware and firmware design for the control rig, and the software components implemented in the VR application. Subsequent sections describe the prototype development process, testing methodology, evaluation metrics, and experimental results. The report concludes with a discussion of limitations, future work, and potential routes for commercialization and IP protection.

