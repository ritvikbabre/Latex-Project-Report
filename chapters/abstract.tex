\chapter*{Abstract}
\addcontentsline{toc}{chapter}{Abstract}

Traditional driver training methods face significant limitations in terms of safety, cost, and scalability, particularly when training on high-end vehicles or in hazardous scenarios. This project proposes the design and development of a high-fidelity Virtual Reality (VR) driving simulator for driving schools, leveraging the Meta Quest 3 platform and custom Bluetooth-enabled physical controls. The proposed system addresses critical research gaps identified in current VR driving simulators: lack of affordable high-fidelity systems with realistic haptic feedback, absence of wireless control integration, limited dual-mode transmission support (automatic and manual), and insufficient validation for educational applications. The simulator integrates custom-built steering wheel, pedal assembly, and gear shifter hardware connected via ESP32 Bluetooth microcontroller, providing realistic tactile feedback while maintaining system cost below \$5,000. The software architecture utilizes Unity 6 game engine to deliver photorealistic vehicle models, accurate physics simulation, and immersive spatial audio at 90+ FPS. The system is designed to support both standalone Quest 3 operation and PC-tethered mode, with comprehensive performance logging and automated assessment capabilities aligned with RTO certification standards. This paper presents the literature review, gap analysis, system requirements, proposed architecture, and design rationale that justify technical decisions. The proposed simulator aims to provide driving schools with a scalable, safe, and cost-effective training platform capable of simulating diverse scenarios including urban driving, highway navigation, emergency response, and luxury vehicle operation that would otherwise be impractical in traditional training environments.

\textbf{Keywords:} Virtual Reality, Driving Simulator, Driver Training, Meta Quest 3, ESP32, Bluetooth Controls, Unity 6, Driver Education, Autonomous Vehicles, Haptic Feedback
